%%%%%%%%%%%%%%%%%%%%%%%%%%%%%%%%%%%%%%%%%%%%%%%%%%%%%%%%%%%%%%%%%%%%%%%%%%%%%%%%%%%%%%%%%%%%
%%
%% Chapter 1 : Introduction
%%
%%      * Should give an intro/overview of the problem and the context in which it arises
%%      * Then, it should emphasize the problem (problem statement)
%%      * Then the objectives of the proposed research topic and problem
%%      * And finally the organization/structure of the document
%%
%%%%%%%%%%%%%%%%%%%%%%%%%%%%%%%%%%%%%%%%%%%%%%%%%%%%%%%%%%%%%%%%%%%%%%%%%%%%%%%%%%%%%%%%%%%%
\chapter{Introduction}
\label{ch:intro}

%%%%%%%%%%%%%%%%%%%%%%%%%%%%%%%
%   Figures for chapter 1
%%%%%%%%%%%%%%%%%%%%%%%%%%%%%%%

\newcommand{\figDrlLocomotionMotivation}{
    \begin{figure}[!ht]
        \centering
        \begin{subfigure}[b]{0.45\textwidth}
            \centering
            \includegraphics[width=0.9\textwidth]{./chapters/chapter_1/imgs/img_ch1_emergence_of_locomotion.png}
            \caption{}
            \label{fig:ch1_emergence_locomotion}
        \end{subfigure}
        \begin{subfigure}[b]{0.45\textwidth}
            \centering
            \includegraphics[width=0.9\textwidth]{./chapters/chapter_1/imgs/img_ch1_deepmimic.png}
            \caption{}
            \label{fig:ch1_deepmimic}
        \end{subfigure}
        \caption{Some results from applying Deep Reinforcement Learning to locomotion. 
                    a) Biped traversing simulated environment \citep{DeepmindEmergenceLocomotion}.
                    b) Humanoid traversing simulated environment \citep{DeepMimic} }
        \label{fig:ch1_motivation_drl_locomotion}
    \end{figure}
}

Recent advances in the field of Deep Reinforcement Learning (DRL) have achieved impressive
results in various complex tasks, like being able to play Atari games at a super human level \citep{DQNAtari}, 
and beating the world's Go champion \citep{AlphaGo}, just to name a few. These represent discrete 
action spaces problems, in which the action space can be defined by a finite-size set of values. 
More recently, this approach has also been applied to continuous actions spaces, i.e. in continuous 
control tasks. This has given good results in various locomotion benchmarks, like in 
\citeauthor{DeepmindEmergenceLocomotion} and \citeauthor{DeepMimic}, shown in Figure @FIG-1.1.

\figDrlLocomotionMotivation

This approach is very promising because it allows an agent to learn an appropiate controller for a given task after training in
the simulated environment, which reduces the need to implement highly sophisticated control pipelines. Currenttly available 
training environments have been presented in various articles, and we will talk more about these in Chapter 3.
Some of these currently available training environments are shown in Figure @FIG-1.2.

%% and compared to the ones mentioned previously (from Figure @FIG-1.1)
%%they are much simpler, consisting mostly of classic control problems and some locomotion tasks in flat terrains.

The ultimate goal of applying these techniques is to be able to develop complex behaviours, similar
to those shown in nature by animals. However, to develop and test this complex behaviours there is a need to \textbf{create
a wide set of rich and complex environments}, which is not provided by currently available locomotion benchmarks.

\figDrlBenchmarks

% @TAG: Refer to content above and "formalize" a bit more
\section{Problem Statement}
\label{sec:problem}

As explained previously, currently there are no robot locomotion benchmarks that allow
to \textbf{create diverse learning environments}, which limits the range of behaviours that can be learned
from scratch using Deep Reinforcement Learning techniques. There are also some other problems that arise in the
context of learning locomotion beavirous:

\begin{itemize}
	\item \textbf{Exploiting} the simulated environment's dynamics: 
			This first problem is caused by the current objective required from reinforcement learning agents: try to maximize
			a cumulative reward over time. This forces the agent to extract the most reward from the environment, which could in
			turn lead to the agent exploiting the dynamics of the environment in a way that it could still solve the task defined by the reward function,
			but using wierd beharious from local minima. This is specially true when designing a reward function. The reward is the signal
			that the agent gets to improve its behaviours, so poorly engineered reward functions could lead to sub-optimal and even bad performance.
	\item \textbf{Inability to transfer} learned  policies into the real world (reality gap):
			The second problem is a direct consequence of the first one. As the agent exploits the 
			dynamics of the environment, the agent will overfit to the given environment, and will 
			not generalize when transferred to the real world as the dynamics are different.
\end{itemize}

These issues arise in the context of the current benchmarks being used to train DRL agents, which are not complex enough to
force the agent to avoid exploiting non-general policies, or to compensate for different (or event unmodeled) dynamics of the real world.
So, to solve part of this issue we need to build complex environments that try to account for the two already mentioned problems, like the ones
in \citeauthor{GoogleBrainSim2Real} and \citeauthor{OpenAISim2real}. These try to account for unmodeled dynamics and generalization by
using randomizatio over the simulated environment and other techniques to force the learning agent not to overfit to a specific behaviour that
only exploites some of the dynamics of the environment.

\figEnvManipSimToreal

% @TAG: Refer to initial content and "emphazise" the objectives
\section{Objectives}
\label{sec:objectives}

\subsection*{General Objective}
Make a framework to build diverse learning environments for robot locomotion tasks, and 
evaluate the performance and generalization capabilities of current state of the art Deep 
Reinforcement Learning algorithms.

\subsection*{Specific Objectives}
\begin{itemize}
 \item Implement software tools on top of current Reinforcement Learning APIs and physics engines 
       to create complex RL tasks for continuous control.
 \item Evaluate the performance of state of the art Deep Reinforcement Learning algorithms in these environments.
\end{itemize}

\figEnvironmentsProposalFromTo

\section{Organization}
\label{sec:organization}

The structure of this proposal is as follows :

\begin{itemize}
	\item @TODO: Overview chap2
	\item @TODO: Overview chap3
	\item @TODO: Overview chap4
	\item @TODO: Overview chap5
\end{itemize}