%%%%%%%%%%%%%%%%%%%%%%%%%%%%%%%%%%%%%%%%%%%%%%%%%%%%%%%%%%%%%%%%%%%%%%%%%%%%%%%%%%%%%%%%%%%%
%%
%% Chapter 5 : Current progress
%%
%%      * Should describe the current state we are in the implementations, research, etc.
%%
%%%%%%%%%%%%%%%%%%%%%%%%%%%%%%%%%%%%%%%%%%%%%%%%%%%%%%%%%%%%%%%%%%%%%%%%%%%%%%%%%%%%%%%%%%%%

\chapter{Current progress}
\label{ch:current_progress}

%%%%%%%%%%%%%%%%%%%%%%%%%%%%%%%
%   Figures for chapter 5
%%%%%%%%%%%%%%%%%%%%%%%%%%%%%%%

\newcommand{\figSupportedAgents}{
    \begin{figure}[!ht]
        \centering
        \begin{subfigure}[b]{0.32\textwidth}
            \centering
            \includegraphics[width=1.0\textwidth]{./chapters/chapter_5/imgs/img_ch5_agents_controlsuite_walker.png}
            \caption{}
        \end{subfigure}
        \begin{subfigure}[b]{0.32\textwidth}
            \centering
            \includegraphics[width=1.0\textwidth]{./chapters/chapter_5/imgs/img_ch5_agents_controlsuite_humanoid.png}
            \caption{}
        \end{subfigure}
        \begin{subfigure}[b]{0.32\textwidth}
            \centering
            \includegraphics[width=1.0\textwidth]{./chapters/chapter_5/imgs/img_ch5_agents_pybullet_laikago.png}
            \caption{}
        \end{subfigure}
        \caption{Current agents supported in the framework. Walker (a) and humanoid (b) 
                 from \citeauthor{Controlsuite}, and laikago (c) from \citeauthor{PyBullet}.}
        \label{fig:ch5_current_supported_agents}
    \end{figure}
}

\newcommand{\figSupportedTerrain}{
    \begin{figure}[!ht]
        \centering
        \includegraphics[width=0.9\textwidth]{./chapters/chapter_5/imgs/img_ch5_terrains_sample.png}
        \caption{Terrain generators that are currently supported in the framework: 
                 variable height using profiles, and obstacle course similar to
                 the terrain used in \citeauthor{DeepmindEmergenceLocomotion}.}
        \label{fig:ch5_current_supported_terrain}
    \end{figure}
}

\newcommand{\figSupportedSensors}{
    \begin{figure}[!ht]
        \centering
        \begin{subfigure}[b]{0.45\textwidth}
        \centering
            \includegraphics[width=1.0\textwidth]{./chapters/chapter_5/imgs/img_ch5_sensors_1.png}
            \caption{}
        \end{subfigure}
        \begin{subfigure}[b]{0.45\textwidth}
            \centering
            \includegraphics[width=1.0\textwidth]{./chapters/chapter_5/imgs/img_ch5_sensors_2.png}
            \caption{}
        \end{subfigure}
        \caption{Current sensors supported in the framework. a) Intrinsic measurements
                 from the joints and bodies of the agent. b) Extrinsic measurements
                 from the terrain using heightfields.}
        \label{fig:ch5_current_supported_sensors}
    \end{figure}
}

\newcommand{\figSupportedVisualizers}{
    \begin{figure}[H]
        \centering
        \begin{subfigure}[b]{0.9\textwidth}
        \centering
            \includegraphics[width=1.0\textwidth]{./chapters/chapter_5/imgs/img_ch5_visualizer_support_1.png}
            \caption{}
        \end{subfigure}
        \begin{subfigure}[b]{0.9\textwidth}
            \centering
            \includegraphics[width=1.0\textwidth]{./chapters/chapter_5/imgs/img_ch5_visualizer_support_2.png}
            \caption{}
        \end{subfigure}
        \caption{Current visualizers supported in the framework. 
                    a) Custom visualizer. 
                    b) MuJoCo visualizer.}
        \label{fig:ch5_current_supported_visualizers}
    \end{figure}
}

\newcommand{\figProgressAgents}{
    \begin{figure}
        \centering
        \includegraphics[width=0.9\textwidth]{./chapters/chapter_5/imgs/img_tysocmjc_agents.png}
        \caption{Current agent support. Refer to \href{https://youtu.be/5zv5SK0o92I}{this}
                 video for a test of the current functionality.}
        \label{fig:ch5_progress_agents}
    \end{figure}
}

\newcommand{\figProgressSensors}{
    \begin{figure}
        \centering
        \includegraphics[width=0.9\textwidth]{./chapters/chapter_5/imgs/img_tysocmjc_sensors.png}
        \caption{Current sensor support. Currently the framework supports: 
                            a) intrinsic measurements (joint angles and velocities, body velocities and acceelrations, and relative positions).
                            b) extrinsic measurements (height fields taken from the terrain)}
        \label{fig:ch5_progress_sensors}
    \end{figure}
}

\newcommand{\figProgressTerrains}{
    \begin{figure}
        \centering
        \includegraphics[width=0.9\textwidth]{./chapters/chapter_5/imgs/img_tysocmjc_terrains.png}
        \caption{Current terrain support. Currently the framework supports the
                 environments from \citeauthor{DeepmindEmergenceLocomotion}}
        \label{fig:ch5_progress_terrain}
    \end{figure}
}

\newcommand{\tableAgentFormatsSupport}{
    \begin{table}[]
        \centering
        \begin{tabular}{c|c|c|c|}
        \cline{2-4}
                                        & \cellcolor[HTML]{C0C0C0}{\color[HTML]{333333} mjcf} & \cellcolor[HTML]{C0C0C0}{\color[HTML]{333333} urdf} & \cellcolor[HTML]{C0C0C0}{\color[HTML]{333333} rlsim} \\ \hline
        \multicolumn{1}{|c|}{Completed} & x                                                   & x                                                   &                                                      \\ \hline
        \multicolumn{1}{|c|}{Tested}    & x                                                   &                                                     &                                                      \\ \hline
        \end{tabular}
        \caption{Current agent formats supported in the framework}
        \label{tab:ch5_table_supported_agent_formats}
    \end{table}
}

\newcommand{\tableBackendsSupport}{
    \begin{table}[]
        \centering
        \begin{tabular}{c|c|c|}
        \cline{2-3}
                                        & \cellcolor[HTML]{C0C0C0}{\color[HTML]{333333} MuJoCo} & \cellcolor[HTML]{C0C0C0}{\color[HTML]{333333} Bullet} \\ \hline
        \multicolumn{1}{|c|}{Completed} & x                                                     &                                                       \\ \hline
        \multicolumn{1}{|c|}{Tested}    & x                                                     &                                                       \\ \hline
        \end{tabular}
        \caption{Current backends supported in the framework}
        \label{tab:ch5_table_supported_backends}
    \end{table}
}

In this chapter we discuss the current progress of the proposed framework. 
The implementation is still not in version 0.1, which should be the first 
fully functional version to be released according to the features proposed 
in the previous chapter. Nevertheless, the core functionality, which is related
to the abstract representations (section ~\ref{subsec:ch4_core_functionality}),
is working and almost completed. There is also a working implementation for
MuJoCo, one of the two backends that we will support. All the implementations 
are available in github, in the following repositories:

\begin{itemize}
    \item \textbf{Core functionality}: the repository that serves as the
          core dependency for the framework. It can be found \href{https://github.com/wpumacay/tysocCore}{here}.
    \item \textbf{MuJoCo implementation}: the repository with the
          adapter code for MuJoCo as the specific backend. It can be found
          \href{https://github.com/wpumacay/tysocMjc}{here}.
\end{itemize}

\section{Core functionality}

As previously mentioned we already have a working implementation of the core
functionality supporting part of the proposed agent formats, and the appropriate
abstractions for the terrain and sensors. This functionality is decoupled of the 
specific backend we used for instantiating the simulation. The key components currently 
available are the following:

\begin{itemize}
    \item \textbf{Agent core functionality}: We implemented the core kinematic 
            tree (explained previously in Figure ~\ref{fig:ch4_core_agent_functionality}), 
            and made sure that this implementation was decoupled from the backend by
            following the bridge approach mentioned in section ~\ref{fig:ch4_bridge_pattern}.
            We currently have support for the \textbf{mjcf} and \textbf{urdf} agent 
            formats. We have also support for three agents from the proposed agents list:
            walker and humanoid from Controlsuite, and laikago from PyBullet. These
            are shown in the Figure below.

            \figSupportedAgents

    \item \textbf{Terrain core functionality}: We implemented the abstract terrain
           functionality (explained previously in Figure ~\ref{fig:ch4_core_terrain_functionality}),
           and also two simple procedural terrain generators: one for forward running
           tasks with terrain generated using boxes as primitives, similar to the
           terrain used in \cite{DeepmindEmergenceLocomotion}, and the other also
           for forward running tasks using variable height terrain generated from a
           profile, connected by boxes as primitives.

           \figSupportedTerrain
    \newpage
    \item \textbf{Sensors core functionality}: We implemented the core sensors
          functionality and made sure that this implementation was decoupled from
          the backend, following the bridge approach mentioned in section ~\ref{fig:ch4_bridge_pattern}.
          The sensors currently supported by the framework consist of intrinsic
          measurements from joints and bodies, and extrinsic readings from the
          terrain using heightfields.

          \figSupportedSensors

\end{itemize}

%% @TODO: talk about the visualizer, and the two implemented options.
%% @TODO: make a table to list done and todos in this feature

%% @TODO: talk about the backends supported so far
%% @TODO: make a table to list done and todos in this feature

\section{Simulation features}

As previously mentioned, the current backend that the framework supports is the
MuJoCo physics engine. The implementation was made with the bridge approach shown
in Figure ~\ref{fig:ch4_bridge_pattern}, and extra care was taken to ensure not
to couple backend specific features in the core framework. The support for this 
current backend is shown in the next sections.

\subsection*{Agents support}

The implementation of the agent's core functionality has been completed to a 80\%,
remaining only an additional implementation of the agents format provided by \cite{TerrainRLSim}.
This implementation includes the agent's kinematic tree implementation and the integration 
with the MuJoCo physics engine via an adapter written for this specific engine.

The core implementation remains agnostic of the specific engine, which allows us
to make support quickly for the remaining proposed physics engines via adapter code
written once for each platform. The current supported agents are shown in Figure ~\ref{fig:ch5_progress_agents},
which basically shows the support for agents made for MuJoCo via its XML representation format.

The current implementation can be already used for experiments from C/C++. The exposed
functionality includes the supported functionality exposed by most benchmarks, which
give access to the actions via torques. The remaining part of the implementation of the core
agent functionality should give more support for different types of actuation models, like in \cite{ActuationChoice}.

\figProgressAgents

\subsection*{Sensors support}

The implementation of the sensors' core functionality is completed to a 70\%,
remaining some sensors for more complicated tasks: visual inputs from a fixed camera,
and heightmap measurements from more complicated terrains. The current supported sensor
features shown in Figure ~\ref{fig:ch5_progress_sensors}, and consist of:

\begin{itemize}
    \item Intrinsic sensor measurements, like joints angles and velocities, bodies velocities
          and accelerations, and bodies relative positions (to the root body).
    \item Extrinsic sensor measurements, which consist on heightfield data from the terrain ahead,
          taken with respect to the root body.
\end{itemize}

\figProgressSensors

\subsection*{Terrain support}

The current implementation of the core terrain functionality supports procedurally generated
terrain, similar to \cite{DeepmindEmergenceLocomotion}. The core functionality handles 
the creation of primitives from terrain generators, and then this is stored as 
requests for creation in the specific adapters for specific engines. The current 
supported engine is MuJoCo, and we plan in adding full support for more complex 
terrains (and an API to use it), and full support for the remaining proposed physics 
engines. Some screenshots of the current terrain implementations are shown
in Figure ~\ref{fig:ch5_progress_terrain}.

\figProgressTerrains

%% @TODO: Add a discussion here
\todo{deberías cerrar el capítulo con una discus ión breve de lo que se hizo}

\tableAgentFormatsSupport
\tableBackendsSupport